\section{The postulates}
We shall start with a general overview of the basic postulates of quantum mechanics. These postulates provide a connection between the physical world and the mathematical formalism of quantum mechanics upon quantum computation is built on.
\begin{postulate}\label{postulate:1}
Associated to any isolated physical system is a Hilbert space $\mathcal{H}$ known as the \emph{state space} of the
system. The system is completely described by its \emph{state vector} $\ket{\psi(t)}$, which is a unit
vector in the system’s state space.
\end{postulate}
\begin{postulate}\label{postulate:2}
The time-evolution of the state of a closed quantum system is described by a
unitary operator. That is, for any evolution of the closed system there exists
a unitary operator $\widehat{U}(t_2,t_1)$ such that if the initial state of the system is $\ket{\psi(t_1)}$ then
after the evolution the state of the system will be:
\begin{equation*}
    \ket{\psi(t_2)} = \hat{U}(t_2,t_1) \ket{\psi(t_1)} \quad \text{with} \quad \hat{U}^\dagger\hat{U} = \hat{\mathbb{I}}\,.
\end{equation*}
\end{postulate}
\begin{postulate}
Quantum measurements are described by a collection $\{M_m\}$ of
measurement operators. These are operators acting on the state space of the
system being measured. The index $m$ refers to the measurement outcomes that
may occur in the experiment. If the state of the quantum system is $\ket{\psi(t)}$
immediately before the measurement then the probability that result m occurs is given by
\begin{equation*}
    p(m) = \expval{M^\dagger_m M_m}{\psi}
\end{equation*}
and the state of the system after the measurement is
\begin{equation*}
    \frac{M_m\ket{\psi}}{\expval{M^\dagger_m M_m}{\psi}}
\end{equation*}
The measurement operators satisfy the \emph{completeness equation}
\begin{equation*}
    \sum_m M_m^\dagger M_m = \mathbb{I}
\end{equation*}
which express the fact that probabilities sum to one: $\sum_n p(m) = \sum_n \expval{M^\dagger_m M_m}{\psi} = 1$.
\end{postulate}
\begin{postulate}\label{postulate:4}
The state space of a composite physical system is the tensor product
of the state spaces of the component physical systems. If we have systems numbered $1$ through $N$:
\begin{equation*}
   \mathcal{H} = \bigotimes_{i=1}^{N}\,\mathcal{H}_i \quad  \text{and} \quad \text{dim}\mathcal{H} = \prod_{i=1}^N \text{dim}\mathcal{H}_i\,.
\end{equation*}
\end{postulate}