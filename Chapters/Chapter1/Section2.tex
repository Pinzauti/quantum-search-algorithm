\section{Quantum bit}
The definition of qubit (quantum bit) immediately follows from \hyperref[postulate:1]{postulate 1}:
\begin{defn}
A \emph{qubit} is a physical system $Q$ whose Hilbert space $\mathcal{H}_Q$ has dimension $\dim\mathcal{H}_S = 2$.
\end{defn}
Because of \hyperref[postulate:1]{postulate 1} and according to the definition of vector space we see that every linear combination of a state vector
\begin{equation}\label{eq:linear-combination}
    \bra{\psi} = a \bra{\alpha} + b \bra{\beta} \quad \bra{\alpha},\bra{\beta} \in \mathcal{H}_\psi \quad a,b \in \mathbb{C} \quad \abs{a}^2 + \abs{b}^2 = 1
\end{equation}
is still part of the state space and it still describes the physics of the system. (There is only a constraint: the state has to be normalized according to \eqref{eq:normalization}, such a rescaling is possible and will be assumed hereafter.) 


Here lies the main difference between bits and qubits: whereas in classical computation only $0$ and $1$ states are allowed, in quantum computation also superposition states are perfectly acceptable. \graffito{Main difference between bits and qubits.} What does a superposition state physically mean? If we measure for example \eqref{eq:linear-combination} the probability of being in the state $\ket{\alpha}$ is  $\abs{a}^2$ and the probability of being in the state $\ket{\beta}$ is $\abs{b}^2$.

According (again) to the \hyperref[postulate:1]{first postulate} the state of a qubit is a vector in a two-dimensional Hilbert space. Let us define its basis:
\begin{defn}\label{def:computational-basis}
The orthonormal basis of the two-dimensional Hilbert space describing a qubit is called \emph{computational basis} and it's composed of the states $\ket{0}$ and $\ket{1}$ known as \emph{computational basis states}.
\end{defn}
How a qubit is physically made?
It can be a $1/2$ spin particle, an atomic system whose dynamics is described by two (non-degenerate) energy levels and so on.
Whatever we choose to be our physical realization of the qubit we have a Hermitian operator associated with the observable chosen. The computational basis then will be composed by the eigenstates of the Hermitian operator associated with the observable\footnote{To every Hermitian operator $\Omega$ defined on a space $\mathcal{H}$ there exist (at least) a basis of the space $\mathcal{H}$ consisting of the orthonormal eigenvectors of the operator. \cite[p.36]{Shankar}.}, those state (whatever the operator is) will be labelled as $\{\ket{0},\ket{1}\}$ according to \hyperref[def:computational-basis]{definition 2}.

\graffito{An example of how a qubit can be physically implemented.}
Let us use, for example, a $1/2$ spin particle.  We know that the Hermitian operator associated with spin is $S_z$ or $\hat{\sigma}_z = \frac{2}{\hbar} S_z$ that fits our scope because it has two eigenstates and two non-degenerate eigenvalues: 

\begin{equation}\label{eq:spin-particle}
\begin{split}
    \hat{\sigma}_z \ket{\chi_+} &= \ket{\chi_+}  \\
    \hat{\sigma}_z \ket{\chi_-} &= -\ket{\chi_-}\,.
\end{split}
\end{equation}
These eigenstates (i.e., the spinors) span a two-dimensional Hilbert space and can be chosen as the computational basis.

\subsection{Quantum register}
The definition of quantum register, quantum analogue of the classical register, immediately follows from \hyperref[postulate:4]{postulate 4}:
\begin{defn}
A $n$ size \emph{quantum register} is a system QR  with $\dim\mathcal{H}_{QR} = 2^n$.
\end{defn}
In other words, a quantum register is a system comprising multiple qubits.


The simplest case is a system $N=2$ with two qubits $Q_1$ and $Q_2$. If we define the basis of $\mathcal{H}_{Q_1}$ and $\mathcal{H}_{Q_2}$ as $\{\ket{0}_1, \ket{1}_1\}$ and $\{\ket{0}_2, \ket{1}_2\}$ the basis of $H_{QR}$ is \graffito{Quantum register with two qubits.}
\begin{equation}\label{eq:quantum-register-basis}
    \{\ket{0}_1 \otimes \ket{0}_2, \ket{1}_1 \otimes \ket{0}_2, \ket{0}_1 \otimes \ket{1}_2, \ket{1}_1 \otimes \ket{1}_2\}\
\end{equation}
and as expected we have $\dim\mathcal{H}_{QR} = 4$.
\subsection{Entanglement}
Consider two arbitrary quantum systems $Q_1$ and $Q_2$, with respective Hilbert spaces $\mathcal{H}_{Q_1}$ and $\mathcal{H}_{Q_2}.$ The Hilbert space of the composite system is the tensor product: 
\begin{equation*}
\mathcal{H}_{Q_1} \otimes \mathcal{H}_{Q_2}\,,
\end{equation*}
if the first system is in state $\ket{\psi}_{Q_1}$ and the second in state $\ket{\psi}_{Q_2}$ the state of the composite system is
\begin{equation*}
    \ket{\psi}_{Q_1} \otimes \ket{\psi}_{Q_2}\,.
\end{equation*}
States of the composite system that can be represented in this form are called \emph{separable states} while
\begin{defn}
A composite system such that $\ket{QR} \neq \otimes_i \ket{Q_i}$ is an \emph{entangled state}.
\end{defn}

If we consider two qubits:
\graffito{An example of a separable state.}
\begin{align*}
    &\ket{Q_1} = a \ket{0}_1 + b \ket{1}_1 \quad a, b \in \mathbb{C}, \quad \abs{a}^2 + \abs{b}^2 = 1 \\
    &\ket{Q_2} = c\ket{0}_2 + d\ket{1}_2 \quad c,d \in \mathbb{C}, \quad \abs{c}^2 + \abs{d}^2 = 1
\end{align*}
the overall state of the system is:
\begin{multline*}
    \ket{QR} = \ket{Q_1} \otimes \ket{Q_2} = ac \ket{0}_1 \otimes \ket{0}_2 \\ +  bc  \ket{1}_1 \otimes \ket{0}_2 + 
    ad \ket{0}_1 \otimes \ket{1}_2 + bd \ket{1}_1 \otimes \ket{1}_2
\end{multline*}
that is a separable state.

If instead we have:
\graffito{An example of an entangled state.}
\begin{equation*}
    \ket{\Phi^+} = \frac{\ket{0}_1 \otimes \ket{0}_2 + \ket{1}_1 \otimes \ket{1}_2}{\sqrt{2}}
\end{equation*}
we immediately see that this is an entangled state\footnote{It is one of the Bell states, four specific maximally entangled quantum states of two qubits.} as there is no way of writing it as $\ket{Q_1} \otimes \ket{Q_2}$. 