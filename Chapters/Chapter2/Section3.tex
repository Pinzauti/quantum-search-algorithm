\section{Algebraic proof of correctness}
We have shown that during the entire computation, the state of the algorithm is a linear combination of $\ket{\psi}$ and $\ket{\beta}$. We can write the action of $\hat{U_\psi}$ and $\hat{U}_\beta$ in the space spanned by $\{\ket{\beta}, \ket{\psi}\}$ as:
\begin{align*}
    &\hat{U}_\psi : a\ket{\beta} + b\ket{\psi} \rightarrow [\ket{\beta}\ket{\psi}] \begin{bmatrix}
    -1 & 0 \\
    2/\sqrt{N} & 1
    \end{bmatrix} 
    \begin{bmatrix}
    a \\
    b
    \end{bmatrix} \\
    &\hat{U}_\beta : a\ket{\beta} + b\ket{\psi} \rightarrow [\ket{\beta}\ket{\psi}] \begin{bmatrix}
    -1 & -2/\sqrt{N} \\
    0 & 1
    \end{bmatrix} 
    \begin{bmatrix}
    a \\
    b
    \end{bmatrix} 
\end{align*}
In the basis $\{\ket{\beta}, \ket{\psi}\}$ the operator $\hat{U_\psi}\hat{U}_\beta$ is given by the matrix:
\begin{equation*}
    \hat{U}_\psi\hat{U}_\beta = \begin{bmatrix}
    -1 & 0 \\
    2/\sqrt{N} & 1
    \end{bmatrix} 
    \begin{bmatrix}
    -1 & -2/\sqrt{N} \\
    0 & 1
    \end{bmatrix}  =
    \begin{bmatrix}
    1 & 2/\sqrt{N} \\
    -2/\sqrt{N} & 1-4\sqrt{N}
    \end{bmatrix} 
\end{equation*}
The matrix, defining $t=\arcsin{\frac{1}{\sqrt{N}}}$, has the following Jordan form
\begin{equation*}
    \hat{U}_\psi\hat{U}_\beta = M  \begin{bmatrix}
    e^{2it} & 0 \\
    0 & e^{-2it}
    \end{bmatrix}  M^{-1} \quad \text{where} \quad M = \begin{bmatrix}
    -i & i \\
    e^{it} & e^{-it}
    \end{bmatrix}.
  \end{equation*}
 With $M$ being the matrix diagonalizing the operators. It follows that after R iterations we have
\begin{equation*}
    (\hat{U_\psi}\hat{U}_\beta)^r = M  \begin{bmatrix}
    e^{2Rit} & 0 \\
    0 & e^{-2Rit}
    \end{bmatrix}  M^{-1}.
\end{equation*}
We can now use trigonometric identities to copmute the probability of observing the result $\beta$ after R iterations.

\textbf{da qui in poi passaggi non chiari}.