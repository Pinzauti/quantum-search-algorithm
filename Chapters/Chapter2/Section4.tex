\section{Computational complexity}
Let's define some notation:

Any computational problem that can be solved by a classical computer can also be solved by a quantum computer \cite[29]{NielsenChuang}. Conversely, any problem that can be solved by a quantum computer can also be solved by a classical computer, at least in principle given enough time.
\subsection{Performance}\label{sec:performance}
Following the geometric interpretation of section~\ref{sec:geometric-interpretation} we are going to calculate the number of iterations $R$ necessary for the algorithm to find the solution. The initial state of the system is~\ref{eq:initial-state} so rotating through $\arccos{\sqrt{M/N}}$ radians to take the system to $\ket{\beta}$.
\begin{defn}
$\text{CI}(x)$ denote the integer closest to the real number $x$, rounding halves down.
\end{defn}
Then repeating the Grover iteration:
\begin{equation*}
    R = \text{CI}\biggl(\frac{\arccos{M/N}}{\theta}\biggr)
\end{equation*}
times rotates $\ket{\psi}$ to within an angle $\theta/2$
\begin{equation*}
    R \leq \frac{\pi}{4} \sqrt{\frac{N}{M}}.
\end{equation*}
\subsubsection{A two bit example}

\subsection{Optimality of the search algorithm}
We shall show that no quantum algorithm can perform the task of searching trhough $N$ unsorted items using fewer than $\Omega(\sqrt{n})$ access to the search oracle.
\begin{theorem}
The quantum search algorithm is optimal. 
\end{theorem}
\begin{proof}
d
\end{proof}
\subsection{NP vs BPQ}