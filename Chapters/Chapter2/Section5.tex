\section{Practical applications}
\subsection{Cryptography}
The quantum search algorithm can be seen as an algorithm for function inversion. If we have a function $y=f(x)$ that can be evaluated on a quantum computer, the quantum search algorithm allow us to calculate $x$ when given $y$.

Consequently, Grover's algorithm gives broad asymptotic speed-ups to many kinds of brute-force attacks on symmetric-key cryptography, including collision attacks\footnote{A collision attack on a cryptographic hash is a brute-force method that tries to find two inputs producing the same hash value.} and pre-image attacks\footnote{A pre-image attack on cryptographic hash functions is is a brute-force method that tries to find a message that has a specific hash value.}.
The algorithm could brute-force a $128$-bit symmetric cryptographic key in $R = \mathbb{O} \biggl(\sqrt{\frac{N}{M}}\biggr) = \mathbb{O} (\sqrt{N}) = \mathbb{O} (\sqrt{2^{128}}) = \mathbb{O} (2^{64})$ iterations and a 256-bit key (the actual standard for cryptography) in $\mathbb{O} (2^{128})$ iterations. Because of that  it is sometimes suggested~\cite{10.1007/978-3-642-12929-2_6} that symmetric key lengths be doubled to protect against future quantum attacks.

\subsection{Limitations}
Qui vorrei parlare dei limiti nella ricerca sui database installati su memorie classiche.